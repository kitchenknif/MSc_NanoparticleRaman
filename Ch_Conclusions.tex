% !TeX spellcheck = english
% !TEX root = thesis.tex
\section{Conclusions}
    As a result of this project, it has been theoretically predicted and experimentally demonstrated, that Raman scattering
    from crystalline silicon nanoparticles can be enhanced by the Mie-type modes of the nanoparticles, provided that the
    spectral positions of the resonances corresponds to the excitation wavelenghth. The strongest enhancement has beeen shown
    to be from magnetic resonances, because of the field confinement inside the particle and the Q factor of the resonance.
    We have demonstrated an $140$-fold increase in Raman scattering from resontant particles, enhanced on the magnetic dipole
    resonance.

    As part of the project, two methods of fabricating crystalline nanopartilces, based on femtosecond laser ablation,
    nanoparticles were developed: a direct laser-writing technique, allowing for precise patterning of the substrate
    (coated with a thin film of $\alpha$-Si), but lacking good control of nanoparticle size and a less precise technique
    of forward laser transfer of nanoparticle to an arbitrary substrate, with very good control of particle diameter in the
    range of $100-200$nm.
