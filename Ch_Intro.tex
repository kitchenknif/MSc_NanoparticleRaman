% !TeX spellcheck = english
% !TEX root = thesis.tex
\section{Introduction}
\label{ch:Intro}
    \subsection{\hlr{Photonics}}
        \label{sec:Photonics}
            Photonics is a field that studies the properties of light, photons; methods
        of controlling and transmitting light. Most notably is the the subfield of fiber optics,
        with practical applications in the telecommunications industry~--- all modern trans-atlatnic
        and trans-pacific telecom cables are optical fibers. A correlated field, slightly misleadingly
        called silicon photonics (silicon, while being the most popular material, is not the only
        material used), sits between fiber optics and electronics. Silicon photonic devices
        generally are used to convert electronic signals to optical signals and vice versa.

            Most modern developments in silicon photonics are trying to miniaturize photonic
        devices and bring them ``deeper'' into electronic circuits ~--- e.g. to try and
        realize all-optical switching for packet routing [CITATION]. This is complicated by
        problems inherent to traditional silicon photonics~--- large waveguide sizes, slow switching,
        based almost exclusively on heating elements, etc... This is where the field of nanophotonics
        can be very important.

        \cite{krasnok2015towards}
    \subsection{\hlr{Dielectric Nanophotonics}}
        \cite{kuznetsov2012magnetic}
    \subsection{\hlr{Fabrication of Dielectric Nanoparticles}}
            There are several techniques to fabricate dielectric nanoparticles. They can be ordered in terms of level of control over the size
        of the particles and the options of building patterned nanostructures for the nanoparticles. Complexity and cost of the processes generally
        correlate with the level of control provided by the different techniques.
            The main techniques can be grouped into: chemical synthesis, thin-film dewetting, laser ablation-based and lithographic methods.

        \subsubsection{\hlr{Chemical synthesis}}
                Chemical vapor deposition of silicon from disilane gas has been used to fabricate silicon nanoparticles, ($Si_2H_6 \rightarrow 2Si + 3H_2$ at
            high temperatures) achieving polycrystalline spherical particles. Monodisperse colloidal particles have been fabricated from trisilane ($Si_3H_8$)
            at high temperature in n-Hexane. Sizes of the particles were controlled by changing the concentration of the trisilane and the temperature of the
            reaction. The main disadvantages of these types of methods are the porosity and high hydrogen concentration of the particles and necessity of
            further processing if ordered nanostructures are required.

        \subsubsection{\hlr{Thin-film dewetting}}
                Thermal dewetting of thin films can be used to create arrays of nanoparticles, with size and phase controlled by the thickness of the intial
            film and temperature of the process. If fabricated from an unpatterned film, the particles will be unordered~--- producing ordered particles is this
            way requires an additional patterning of the the film, usually by lithographic processes, which increases the complexity of the process and the cost.
        \subsubsection{\hlr{Laser ablation-based methods}}
                Laser ablation by focused ultra-short pulses can also be used to fabricated nanoparticles by heating up the irradiated area to eject material into
            spherical particles deposited either near the irradiated area or transferred to another substrate. Control of the beam sport, fluency and donor material
            structure and be used to control individual particle size and the number of particles fabricated from a single pulse (down to one particle in best-case
            scenario). The ultra-short pulses can also be used to change the phase of already fabricated structures by means of light-induced thermal annealing.
        \subsubsection{\hlr{Lithographic methods}}
                Using electron beam lithography and reactive ion etching, one can pattern substrates with high-quality nanocylinders, with near-perfect
            control of both size and positioning of the structures. This give great flexibility in terms of fabricating ordered nanostructures, but at the
            cost of extreme complexity and high costs.

%            \begin{figure}[h!]
%    				\begin{center}
%    					%\includegraphics[width=0.5\textwidth]{figs/intro/}
%    				\end{center}
%                    \label{fig:Lithography}
%                    \caption{a. Schematic of lithographic process and b. SEM image of possible structure}
%            \end{figure}
\clearpage
