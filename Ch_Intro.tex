% !TeX spellcheck = english
% !TEX root = thesis.tex
\section{Introduction}
\label{ch:Intro}
    \subsection{\hlr{Nanophotonics}}
        \label{sec:Nanophotonics}
            Nanophotonics is a field that studies light manipulation at the nanoscale~--- using various nanostructures to control the
        propagation of light. Nanophotonic devices can range from relatively simple perfect reflectors/absorbers to optical computers.
        Applications include antireflective coatings for solar cells, all-optical switching for optical telecommunications, and various
        medical sensing tasks.

            Traditionally nanophotonic devices have utilized plasmonic nanostructures. Plasmonic metallic nanoparticles have very strong
        electric dipole resonances, meaning that they readily interact with the electric component of electromagnetic fields. Problems arise
        when one tries to control the magnetic component of electromagnetic fields, because plasmonic nanoparticles do not have an inherent
        magnetic dipole response. A solution is the split-ring resonator~--- a nanostructure with a magnetic dipole response. Having building
        blocks to control both the electric and magnetic components of electromagnetic fields, plasmonic structures have been efficiently
        used in frequency ranges from gigahertz to several hundred terahertz~--- up to infrared wavelengths. But, because of strong losses
        in the optical spectral range, and of the complexity of fabricating split-ring resonators for optical wavelengths, plasmonics have
        had a lot of difficulty in achieved the required performance for optical nanophotnoic devcies.

        \cite{krasnok2015towards}
    \subsection{\hlr{Dielectric Nanophotonics}}
            This is where the concept of dielectric nanophotonics comes into play. Mie theory predicts\cite{mie1908beitrage} and
        it has been experimentally demonstrated\cite{kuznetsov2012magnetic} that dielectric nanoparticles have both electric and
        magnetic resonances. The positions of the resonances are size and shape dependant, making them easily tunable for any wavelength.
        Crystalline silicon has proven itself as a good material for dielectric nanophotonics~--- low losses at optical wavelengths\cite{palik1998handbook},
        high refractive index, and compatibility with many fabrication processes\cite{popa2008compact,zhao2009mie,evlyukhin2010optical,garcia2011strong,
        krasnok2012all,ginn2012realizing,fu2012directional,krasnok2015towards}.

            Dielectric nanoparticles have been used for sensing and eletromagnetic field enchancement\cite{albella2013low,zambrana2015purcell,
        bakker2015magnetic,caldarola2015non}, antireflective coatings\cite{spinelli2012broadband},  perfect reflectors\cite{evlyukhin2010optical,
        moitra2014experimental}, light wavefront manipulation\cite{decker2015high,yu2015high}, superdirective scattering\cite{krasnok2014superdirective,
        krasnok2014experimental} and enhancement of nonlinear effects\cite{shcherbakov2014enhanced,makarov2015tuning}.

    \subsection{\hlr{Fabrication of Dielectric Nanoparticles}}
            There are several techniques to fabricate dielectric nanoparticles. They can be ordered in terms of level of control over the size
        of the particles and the options of building patterned nanostructures for the nanoparticles. Complexity and cost of the processes generally
        correlate with the level of control provided by the different techniques.
            The main techniques can be grouped into: chemical synthesis\cite{shi2012new}, thin-film dewetting\cite{abbarchi2014wafer},
        laser ablation-based\cite{zywietz2014laser} and lithographic methods.

        \subsubsection{\hlr{Chemical synthesis}}
                Chemical vapor deposition of silicon from disilane gas has been used to fabricate silicon nanoparticles, ($Si_2H_6 \rightarrow 2Si + 3H_2$ at
            high temperatures) achieving polycrystalline spherical particles. Monodisperse colloidal particles have been fabricated from trisilane ($Si_3H_8$)
            at high temperature in n-Hexane. Sizes of the particles were controlled by changing the concentration of the trisilane and the temperature of the
            reaction. The main disadvantages of these types of methods are the porosity and high hydrogen concentration of the particles and necessity of
            further processing if ordered nanostructures are required.

        \subsubsection{\hlr{Thin-film dewetting}}
                Thermal dewetting of thin films can be used to create arrays of nanoparticles, with size and phase controlled by the thickness of the intial
            film and temperature of the process. If fabricated from an unpatterned film, the particles will be unordered~--- producing ordered particles is this
            way requires an additional patterning of the the film, usually by lithographic processes, which increases the complexity of the process and the cost.
        \subsubsection{\hlr{Laser ablation-based methods}}
                Laser ablation by focused ultra-short pulses can also be used to fabricated nanoparticles by heating up the irradiated area to eject material into
            spherical particles deposited either near the irradiated area or transferred to another substrate. Control of the beam sport, fluency and donor material
            structure and be used to control individual particle size and the number of particles fabricated from a single pulse (down to one particle in best-case
            scenario). The ultra-short pulses can also be used to change the phase of already fabricated structures by means of light-induced thermal annealing.
        \subsubsection{\hlr{Lithographic methods}}
                Using electron beam lithography and reactive ion etching, one can pattern substrates with high-quality nanocylinders, with near-perfect
            control of both size and positioning of the structures. This give great flexibility in terms of fabricating ordered nanostructures, but at the
            cost of extreme complexity and high costs.

%            \begin{figure}[h!]
%    				\begin{center}
%    					%\includegraphics[width=0.5\textwidth]{figs/intro/}
%    				\end{center}
%                    \label{fig:Lithography}
%                    \caption{a. Schematic of lithographic process and b. SEM image of possible structure}
%            \end{figure}
\clearpage
