% !TeX spellcheck = english
% !TEX root = thesis.tex
\section*{Introduction}
\label{ch:Intro}
        Nanophotonics is a field that studies light manipulation at the nanoscale~--- using various nanostructures to control the
    propagation of light. Nanophotonic devices can range from relatively simple perfect reflectors/absorbers to optical computers.
    Applications include antireflective coatings for solar cells, all-optical switching for optical telecommunications, and various
    medical sensing tasks.

        Traditionally nanophotonic devices have utilized plasmonic nanostructures. Plasmonic metallic nanoparticles have very strong
    electric dipole resonances, meaning that they readily interact with the electric component of electromagnetic fields. Problems arise
    when one tries to control the magnetic component of electromagnetic fields, because plasmonic nanoparticles do not have an inherent
    magnetic dipole response. A solution is the split-ring resonator~--- a nanostructure with a magnetic dipole response. Having building
    blocks to control both the electric and magnetic components of electromagnetic fields, plasmonic structures have been efficiently
    used in frequency ranges from gigahertz to several hundred terahertz~--- up to infrared wavelengths. But, because of strong losses
    in the optical spectral range, and of the complexity of fabricating split-ring resonators for optical wavelengths, plasmonics have
    had a lot of difficulty in achieved the required performance for optical nanophotnoic devcies.\cite{krasnok2015towards}

        This is where the concept of dielectric nanophotonics comes into play. Mie theory predicts\cite{mie1908beitrage} and
    it has been experimentally demonstrated\cite{kuznetsov2012magnetic} that dielectric nanoparticles have both electric and
    magnetic resonances. The positions of the resonances are size and shape dependant, making them easily tunable for any wavelength.
    Crystalline silicon has proven itself as a good material for dielectric nanophotonics~--- low losses at optical wavelengths\cite{palik1998handbook},
    high refractive index, and compatibility with many fabrication processes\cite{popa2008compact,zhao2009mie,evlyukhin2010optical,garcia2011strong,
    krasnok2012all,ginn2012realizing,fu2012directional,krasnok2015towards}.

        Dielectric nanoparticles have been used for sensing and eletromagnetic field enchancement\cite{albella2013low,zambrana2015purcell,
    bakker2015magnetic,caldarola2015non}, antireflective coatings\cite{spinelli2012broadband},  perfect reflectors\cite{evlyukhin2010optical,
    moitra2014experimental}, light wavefront manipulation\cite{decker2015high,yu2015high}, superdirective scattering\cite{krasnok2014superdirective,
    krasnok2014experimental} and enhancement of nonlinear effects\cite{shcherbakov2014enhanced,makarov2015tuning}.

\clearpage
