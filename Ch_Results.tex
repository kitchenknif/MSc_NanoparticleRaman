% !TeX spellcheck = english
% !TEX root = thesis.tex
\section*{Results}
        Two methods of fabricating crystalline nanoparticles, based on femtosecond laser ablation, nanoparticles were developed.
    A direct laser-writing technique, allowing for precise patterning of a substrate (coated with a thin film of $\alpha$-Si),
    where a train of femto-second laser pulses is used to pattern a thin film of $\alpha$-Si~--- cutting it into isolated patches
    of silicon, which dewet into nanoparticles from the absorbed heat from the pulses. With precise optimization of the thickness
    of the film, the laser fluence and pulse frequency, it is possible to create ordered arrays of nearly identical nanoparticles.
    The second technique, forward transfer of nanoparticles by single femtosecond laser pulses from an amorphous thin film onto
    a acceptor substrate, is more flexible~--- particle diameter can be controlled in the range of $100-200$nm by adjusting the
    fluence of the laser pulses, there is virtually no limitation on the type of acceptor substrate, i.e. almost anything can be coated
    by crystalline nanoparticles. Both methods produce crystalline nanoparticles in a single-step process, making the fabrication
    methods very simple and potentially high throughput.

    It has also been shown, both theoretically and experimentally, that Raman scattering from the crystalline silicon nanoparticles
    can be enhanced by the Mie-type modes of the nanoparticles. This happens when the wavelength of the excitation source corresponds
    to the wavelength of a Mie resonance of the nanoparticle. The enhancement happens because of field confinement inside the nanoparticle,
    leading to more efficient excitation of the Raman-active volume of crystalline silicon.
    The strongest enhancement has been shown to be from magnetic resonances, because of the strong field confinement inside the particle
    and high Q factor of the resonance.
    $140$-fold increase in Raman scattering from resonant particles, enhanced on the magnetic dipole resonance has been demonstrated.

    \clearpage
    The main results of the project can be summarized as:
    \begin{enumerate}
        \item Developed two methods for single-stage fabrication of crystalline silicon nanoparticles using femto-second laser ablation:
            \begin{itemize}
                \item Direct laser writing of arrays of nanoparticles in an amorphous silicon thin-film
                \item Forward laser transfer of crystalline nanoparticles from amorphous silicon thing-films to arbitrary substrates
            \end{itemize}
        \item Demonstrated crystallinity of nanoparticles by electron diffraction and Raman measurements
        \item Implemented DDA modeling of apsherical nanoparticles, compared shape-dependent shifts of resonances to pure Mie-theory
            and FIT modeling
        \item Measured elastic scattering spectra from single nanoparticles in dark-field configuration
        \item Estimated sizes of nanoparticles based on positions of Mie-type resonances
        \item Measured Raman scattering intensity from crystalline nanoparticles
        \item Demonstrated resonant behavior of Raman signal intensity dependent on relative positions of excitation wavelength and Mie-type resonances
            of the nanoparticles
    \end{enumerate}

        The results of the project were published in the peer-reviewed journal Nanoscale~--- ``Laser fabrication of
    crystalline silicon nanoresonators from an amorphous film for low-loss all-dielectric nanophotonics''~\citeA{dmitriev2016laser} and
    ``Resonant Raman scattering from silicon nanoparticles enhanced by magnetic response''~\citeA{dmitriev2016resonant}.

\clearpage
