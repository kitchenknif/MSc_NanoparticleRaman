% !TeX spellcheck = english
% !TEX root = thesis.tex
\section*{Results}
    The main results of the project are:
    \begin{enumerate}
        \item A single-stage technique of fabricating crystalline nanoparticles out of amorphous
            semiconductor thin-films, using femtosecond laser ablation was developed. The method
            has control over the size of the fabricated particles and transfers the nanoparticles
            to arbitrary substrates.
        \item X-Ray diffractometry and Raman spectroscopy were used to demonstrate that the
            fabricated nanoparticles are in fact crystalline.
        \item Dark-field spectroscopy was used to demonstrate that the nanoparticles have
            ``Mie''-type optical resonances.
        \item Single nanoparticle Raman spectroscopy was used to demonstrate resonant enhancement of Raman scattering
            from the nanoparticles on their magnetic dipole resonances.
        \item The results of the fabrication section of this project were published in the peer-review
            journal Nanoscale~--- ``Laser fabrication of
            crystalline silicon nanoresonators from an amorphous film for low-loss all-dielectric nanophotonics''\cite{dmitriev2016laser}.
        \item The results of the Raman enhancement section of this project were published in the peer-review
            journal Nanoscale~--- ``Resonant Raman scattering from silicon nanoparticles
            enhanced by magnetic response''\cite{dmitriev2016resonant}.
    \end{enumerate}

    As a result of this project, it has been theoretically predicted and experimentally demonstrated, that Raman scattering
    from crystalline silicon nanoparticles can be enhanced by the Mie-type modes of the nanoparticles, provided that the
    spectral positions of the resonances corresponds to the excitation wavelenghth. The strongest enhancement has beeen shown
    to be from magnetic resonances, because of the field confinement inside the particle and the Q factor of the resonance.
    We have demonstrated an $140$-fold increase in Raman scattering from resontant particles, enhanced on the magnetic dipole
    resonance.

    As part of the project, two methods of fabricating crystalline nanopartilces, based on femtosecond laser ablation,
    nanoparticles were developed: a direct laser-writing technique, allowing for precise patterning of the substrate
    (coated with a thin film of $\alpha$-Si), but lacking good control of nanoparticle size and a less precise technique
    of forward laser transfer of nanoparticle to an arbitrary substrate, with very good control of particle diameter in the
    range of $100-200$nm.

\clearpage
