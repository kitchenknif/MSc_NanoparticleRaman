% !TeX spellcheck = english
% !TEX root = thesis.tex
\section*{Results}
    As a result of this project, it has been theoretically predicted and experimentally demonstrated, that Raman scattering
    from crystalline silicon nanoparticles can be enhanced by the Mie-type modes of the nanoparticles, provided that the
    spectral positions of the resonances corresponds to the excitation wavelength. The strongest enhancement has been shown
    to be from magnetic resonances, because of the field confinement inside the particle and the Q factor of the resonance.
    We have demonstrated an $140$-fold increase in Raman scattering from resonant particles, enhanced on the magnetic dipole
    resonance.

    As part of the project, two methods of fabricating crystalline nanoparticles, based on femtosecond laser ablation,
    nanoparticles were developed: a direct laser-writing technique, allowing for precise patterning of the substrate
    (coated with a thin film of $\alpha$-Si), but lacking good control of nanoparticle size and a less precise technique
    of forward laser transfer of nanoparticle to an arbitrary substrate, with very good control of particle diameter in the
    range of $100-200$nm.


    The main results of the project can be summarized as:
    \begin{enumerate}
        \item Developed two methods for single-stage fabrication of crystalline silicon nanoparticles using femto-second laser ablation:
            \begin{itemize}
                \item Direct laser writing of arrays of nanoparticles in an amorphous silicon thin-film
                \item Forward laser transfer of crystalline nanoparticles from amorphous silicon thing-films to arbitrary substrates
            \end{itemize}
        \item Demonstrated crystallinity of nanoparticles by electron diffraction and Raman measurements
        \item Implemented DDA modeling of apsherical nanoparticles, compared shape-dependent shifts of resonances to pure Mie-theory
            and FIT modeling
        \item Measured elastic scattering spectra from single nanoparticles in dark-field configuration
        \item Estimated sizes of nanoparticles based on positions of Mie-type resonances
        \item Measured Raman scattering intensity from crystalline nanoparticles
        \item Demonstrated resonant behavior of Raman signal intensity dependent on relative positions of excitation wavelength and Mie-type resonances
            of the nanoparticles
    \end{enumerate}

        The results of the project were published in the peer-reviewed journal Nanoscale~--- ``Laser fabrication of
    crystalline silicon nanoresonators from an amorphous film for low-loss all-dielectric nanophotonics''\cite{dmitriev2016laser} and
    ``Resonant Raman scattering from silicon nanoparticles enhanced by magnetic response''\cite{dmitriev2016resonant}.

\clearpage
