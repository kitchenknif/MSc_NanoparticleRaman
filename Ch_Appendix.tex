% !TeX spellcheck = english
% !TEX root = thesis.tex
\section{Mie Scattering of Light}
\label{ap:Mie}
    An important problem for dielectric nanophotonics is the scattering of electromagnetic radiation by
    a homogeneous sphere. This problem has an analytical solution, generally called Mie theory\cite{mie1908beitrage}. The following
    is a condensed version of the solution, following the presentation from Ref. \cite{ng2000manipulation}.
    We will assume an x-polarized incident wave with amplitude $E_0$, propagation constant $\beta_0$ traveling in the $z$ direction:

    \begin{align}
        \vec{E}_{inc} = E_0 e^{i\beta_0z}\hat{x}
    \end{align}

    \subsection{Maxwell's Equations}
        Starting with
        \begin{align}
            \nabla \times \vec{E} &= i\omega\mu\vec{H} \\
            \nabla \times \vec{H} &= -i\omega\epsilon\vec{E}
        \end{align}

        Taking the rotor of the equations and substituting,

        \begin{align}
            \nabla \times (\nabla \times \vec{E}) &= i\omega\mu \nabla \vec{H} = \omega^2 \epsilon\mu\vec{E} \\
            \nabla \times (\nabla \times \vec{H}) &= i\omega\epsilon \nabla \vec{E} = \omega^2 \epsilon\mu\vec{H}
        \end{align}

        Applying the vector identity,

        \begin{align}
            \nabla \times \nabla \vec{A} = \nabla (\nabla \cdot \vec{A}) - \nabla \cdot (\nabla\vec{A})
        \end{align}

        We get the following wave equations

        \begin{align}
            \nabla^2\vec{E} + k^2_m\vec{E} &= 0 \label{mie:waveE}\\
            \nabla^2\vec{H} + k^2_m\vec{H} &= 0 \label{mie:waveH}\\
            k^2_m &= \omega^2\epsilon\mu \label{mie:kvec}
        \end{align}

        With $k_m$ as the wave vector in the surrounding medium. The final aim of this derivation is to get vector solutions
        of the wave equations. We begin by
        \begin{itemize}
            \item Transitioning to a spherical coordinate system ${r, \theta, \phi}$, since our system is spherically symmetrical
            \item Defining a scalar function $\psi_{l,m}$
            \item Defining a constant vector $\vec{r}$
        \end{itemize}

        The scalar function will be a solution of
        \begin{align}
            \nabla^2\psi + k^2_m\psi = 0 \label{mie:scalar}
        \end{align}

        We can construct three vector solutions:

        \begin{align}
            \vec{L} &= \nabla\psi_{l,m} \\
            \vec{M}_{l,m} &= \nabla\times\vec{r}\psi_{l,m} \\
            \vec{N}_{l,m} &= \frac{1}{k_m}\nabla\times\vec{M}_{l,m}
        \end{align}

        All solutions satisfy the wave equations. $\vec{N}_{l,m}$ and $\vec{M}_{l,m}$ are solenoidal functions and are rotors of each other,
        like $\vec{H}$ and $\vec{E}$. $\vec{L}$, on the other hand is purely longitudinal, so we omit it in this analysis.

    \subsection{Scalar Solution}

        In spherical coordinates, the scalar solution, $\psi_{l,m}$ of Equation \ref{mie:scalar} is a function of ${R, \theta, \phi}$
        \begin{align}
            \frac{1}{r^2}\frac{\partial}{\partial r}\left(r^2\frac{\partial\psi}{\partial r}\right)
                + \frac{1}{r^2\sin(\theta)}\frac{\partial}{\partial\theta}\left(\sin(\theta\frac{\partial\psi}{\partial\theta})\right)
                + \frac{1}{r^2\sin^2(\theta)}\frac{\partial^2\psi}{\partial\phi^2} + k^2_m\psi = 0
        \end{align}

        Next, we seek a solutions that separates the variables:

        \begin{align}
            \psi(r,\theta,\phi) &= R(r)\Theta(\theta)\Phi(\phi)
        \end{align}

        Defining constants $m, Q$, we separate the components into separate solutions:

        \begin{align}
            \frac{d^2\Phi}{d\phi^2} + m^2\Phi &= 0 \\
            (1 - \cos^2(\theta))\frac{d^2\Theta}{d(\cos(\theta))^2} - 2\cos(\theta)\frac{d\Theta}{d(\cos(\theta))}
                + (Q - \frac{p^2}{1-cos^2(\theta)}) &= 0 \\
            r^2\frac{d^2R}{dr^2} + 2r\frac{dR}{dr} + (k^2_mr^2 - Q^2)R &= 0
        \end{align}

        The solutions to these equations are as follows:

        For $\Phi$
        \begin{align}
            \Phi = e^{\pm i m \phi}
        \end{align}

        For $\Theta$, representing it as an associated Legendre equation:
        \begin{align}
            Q &= l(l+1) \rightarrow \\
            \Theta &= P^m_l(\nu) = \frac{(1-\nu^2)^{\frac{m}{2}}}{2^l l!}\frac{d^{l+m}(\nu^2-1)^l}{d(\nu)^{l+m}} \\
            \nu &= \cos(\theta)
        \end{align}

        From now on, $P_l^m = P_l^m(\nu)$.

        And for $R$
        \begin{align}
            R &= \sqrt{\frac{2}{\pi}}Z_l(p) \\
            p &= k_m r
        \end{align}

        Where $Z_l(p)$ represents the radial spherical Bessel $j_l(p)$ or first order Hankel $h_l(p)$. $h_l(p)$, being infinite in
        the far field are used to represent an outgoing spherical wave pattern for the scattered field. $j_l(p)$ is finite in the
        origin, so it is a correct representation of incident and transmitted fields.

        Combining all of these,

        \begin{align}
            \psi_{l,m}(r, \theta, \phi) = \sqrt{\frac{1}{\pi}}Z_l(k_m r)P_l^m e^{im\phi}
        \end{align}

        or, separating into even and odd components:

        \begin{align}
            \psi_{l,m,\substack{e\\ o}}(r, \theta, \phi) = \sqrt{\frac{1}{\pi}}Z_l(k_m r)P_l^m \substack{\cos\\\sin}(m\phi)
        \end{align}

    \subsection{Vector Solution}

        Using the previous equation,

        \begin{align}
            \vec{M}_{l,m, \substack{e \\ o}} &= \nabla \times \hat{r}(r \psi_{l, m, \substack{e \\ o }})\\
            \vec{r} &= \hat{r}r
        \end{align}

        By applying the rotor:

        \begin{align}
            \vec{M}_{l,m}(\hat{r}) &= 0 \\
            \vec{M}_{l,m, \substack{e \\ o}} &= \frac{1}{r\sin(\theta)}\frac{d(r\psi)}{d\phi}\hat{\theta} - \frac{1}{r}\frac{d(r\psi)}{d\theta}\hat{\phi} \\
            &= \mp Z_l\frac{P^m_l}{\sin(\theta)}\substack{\sin \\\cos}(m\phi)\hat{\theta} - Z_l\frac{dP^m_l}{d\theta}\substack{\cos \\ \sin}(m\phi)\hat{\phi}
        \end{align}

        And for $\vec{N}_{m,l\substack{e\\ o}}$

        \begin{align}
            \vec{N}_{l,m,\substack{e\\ o}} &= \frac{l(l+1)}{k_mr}\psi_{\substack{e\\ o}}\hat{r}
                + \frac{1}{k_mr}\frac{d(r\vec{M}_{l,m,\phi})}{dr} \hat{\theta} + \frac{1}{k_m r}\frac{d(r\vec{M}_{l,m,\theta})}{dr}\hat{\phi} \\
            &= \frac{l(l+1)}{k_mr}Z_l P_l^m \substack{\cos\\\sin}(m\phi) \hat{r}
                + \frac{1}{r}\frac{d(pZ_l)}{dp}\frac{P_l^m}{d\theta}\substack{\cos\\\sin}m\phi\hat{\theta}  \\
            &\mp m\frac{1}{p}\frac{d(pZ_l)}{dr}\frac{P_l^m}{\sin(\theta)}\substack{\sin\\\cos}(m\phi)\hat{\phi}
        \end{align}

        radial $p$ needs to be replaced by $Np$, $N = \frac{N_s}{N_m}$, which is the relative index of the sphere to the surrounding
        medium.


    \subsection{Incident, Scattered and Internal Fields}
        We assume, that an arbitrary wave, expressed by $\vec{A}$ can be represented by a linear combination of vector functions:

        \begin{align}
            \vec{A} = \frac{i}{\omega}\sum_{l,m}\left(A_{l,m}\vec{M}_{l,m}+B_{l,m}\vec{N}_{l,m}\right)
        \end{align}

        Since $\vec{M}_{l,m}$ and $\vec{N}_{l,m}$ are solenoidal function that correspond to interdependence of $\vec{H}$ and $\vec{E}$, using $\vec{A}$:

        \begin{align}
            \vec{H}_{inc} &= \frac{1}{i\omega \mu}\nabla \times \vec{A} \\
            &= - \frac{i}{\omega\mu}\sum\left(A_{l,m}(\nabla\times\vec{M}_{l,m}) + B_{l,m}(\nabla\times\vec{N}_{l,m})\right) \\
            &= - \frac{ik_m}{\omega\mu}\sum\left(A_{l,m}\vec{N}_{l,m}+B_{l,m}\vec{M}_{l,m}\right)
        \end{align}

        Similarly,

        \begin{align}
            \vec{E}_{inc} = \frac{k_m}{\omega^2\epsilon\mu}\sum\left(A_{l,m}\vec{M}_{l,m} + B_{l,m}\vec{N}_{l,m}\right)
        \end{align}

        $A_{l,m}, B_{l,m}$ are expansion coefficients for a particular beam:

        \begin{align}
            A_{l,m} = \int M^*_{l,m}\vec{E}_{inc}d\Omega \\
            B_{l,m} = \int N^*_{l,m}\vec{E}_{inc}d\Omega \\
        \end{align}

        Where $\Omega = 4\pi r$ is the enclosed surface area.

        Similarly, the scattered and internal fields can be expanded in terms of $\vec{M}_{l,m}, \vec{N}_{l,m}$:

        \begin{align}
            \vec{E}_{scat}=& \frac{k_m}{\omega^2\epsilon\mu}\sum\left(A_{l,m}a_l\vec{M}_{l,m} + B_{l,m}b_l\vec{N}_{l,m}\right)\\
            \vec{H}_{scat}=&-\frac{k_m}{\omega\mu}\sum\left(A_{l,m}a_l\vec{M}_{l,m} + B_{l,m}b_l\vec{N}_{l,m}\right)\\
            \vec{E}_{int}=&\frac{k_m}{\omega^2\epsilon_{int}\mu}\sum\left(A_{l,m}c_l\vec{M}_{l,m} + B_{l,m}d_l\vec{N}_{l,m}\right)\\
            \vec{H}_{int}=-&\frac{ik_m}{\omega\mu}\sum\left(A_{l,m}c_l\vec{M}_{l,m} + B_{l,m}d_l\vec{N}_{l,m}\right)
        \end{align}

        Where $a_l, b_l$ are scattering coefficients and $c_d, d_l$ are internal field coefficients.

    \subsection{Mie Coefficients}

        The Mie coefficients $a_l, b_l, c_l, d_l$ can be determined from boundary conditions on the edge of the sphere.

        \begin{align}
            \left( \vec{E}_{inc} + \vec{E}_{scat} - \vec{E}_{int} \right) \times \vec{r} &= 0 \\
            \left( \vec{H}_{inc} + \vec{H}_{scat} - \vec{H}_{int} \right) \times \vec{r} &= 0
        \end{align}

        or,

        \begin{align}
            E_{inc,\theta} + E_{scat,\theta} &= E_{int, \theta} \\
            E_{inc,\phi} + E_{scat,\phi} &= E_{int, \phi} \\
            H_{inc,\theta} + H_{scat,\theta} &= H_{int, \theta} \\
            H_{inc,\phi} + H_{scat,\phi} &= H_{int, \phi}
        \end{align}

        Which, substituting the vector spherical harmonics, gives:

        \begin{align}
            j_l(N\xi)c_l + h_l(\xi)b_l&=j_l(\xi)\\
            [N\xi j_l(N\xi)]'c_l+ [\xi h_l(\xi)]'b_l&= [\xi j_l(\xi)]'\\
            Nj_l(N\xi)d_l + h_l(\xi)a_l &= j_l(\xi)\\
            [N\xi j_l(N\xi)]'d_l +N[\xi h_l(\xi)]'a_l &= N[\xi j_l(\xi)]'
        \end{align}

        which gives us the the standard expressions for the Mie coefficients:

        \begin{align}
            a_l &= \frac{N^2 j_l(N\xi)[\xi j_l(\xi)]' - j_l(\xi)[N\xi j_l(N \xi)]'}{N^2 j_l(N\xi)[\xi h_l(\xi)]' - h_l(\xi)[N\xi j_l(N \xi)]'}\\
            b_l &= \frac{j_l(N\xi)[\xi j_l(\xi)]' - j_l(\xi)[N\xi j_l(N\xi)]'}{j_l(N\xi)[\xi h_l(\xi)]' - h_l(\xi)[N\xi j_l(N\xi)]'}\\
            c_l &= \frac{j_l(\xi)[\xi h_l(\xi)]' - h_l(\xi)[\xi j_l(\xi)]'}{j_l(N\xi)[\xi h_l(\xi)]' - h_l(\xi)[N\xi j_l(N\xi)]'}\\
            d_l &= \frac{Nj_l(\xi)[\xi h_l(\xi)]' - N h_l(\xi)[\xi j_l(\xi)]'}{N^2 j_l(N\xi)[\xi h_l(\xi)]' - h_l(\xi)[N\xi j_l(N\xi)]'}
        \end{align}

    \subsection{Cross Sections}
        Scattering and extinction cross sections can be easily computed, knowing the Mie coefficients.

        \begin{align}
            C_{sca} &= \frac{W_{sca}}{I_{inc}} \\
            C_{ext} &= \frac{W_{ext}}{I_{inc}}
        \end{align}

        \begin{align}
            W_{sca} &= \frac{1}{2}\int_0^{2\pi}\int_0^\pi \left(E_{sca} \times H^*_{sca}\right)r^2\sin(\theta)d\theta d\phi \\
             &= \frac{1}{2}\int_0^{2\pi}\int_0^\pi \left(E_{sca,\theta} \times H^*_{sca,\phi}\right.
                            \left.- E_{sca,\phi} \times H^*_{sca,\theta}\right)r^2\sin(\theta)d\theta d\phi
        \end{align}

        \begin{align}
            W_{ext} &= \frac{1}{2}\int_0^{2\pi}\int_0^\pi \left(E_{inc} \times H^*_{sca}\right)r^2\sin(\theta)d\theta d\phi \\
             &= \frac{1}{2}\int_0^{2\pi}\int_0^\pi \left(E_{inc,\phi} \times H^*_{sca,\theta}\right.
                            \left.- E_{inc,\theta} \times H^*_{sca,\phi} - E_{sca,\phi} \times H^*_{inc,\theta}
                                           + E_{sca,\theta} \times H^*_{inc,\phi}\right)r^2\sin(\theta)d\theta d\phi
        \end{align}

        Which can be simplified to

        \begin{align}
            C_{sca} &= \frac{2\pi}{k^2_m}\sum_{l=1}^\infty (2l +1)(|a_l|^2 + |b_l|^2)\\
            C_{ext} &= \frac{2\pi}{k^2_m}\Re\sum_{l=1}^\infty (2l +1)(a_l + b_l)\\
            C_{abs} &= C_{ext} - C_{sca}
        \end{align}

\clearpage

\section{Raman Scattering from Crystalline Materials}
\label{ap:Raman}
    Raman scattering of light from crystalline materials is a versatile method of probing the phonon structure of the materials.
    A simplistic classical model of Raman scattering is sufficient to demonstrate the effect and to properly predict many of the
    Raman scattering peaks of semiconductors\cite{peter2010fundamentals}.

    We start with an infinite medium with electric susceptibility $\chi$. For simplicity, let us assume that the medium is isotropic
    and that the susceptibility is scalar. A plane sinusoidal wave is present in the medium, inducing sinusoidal polarization:
    \begin{align}
        \vec{F}(\vec{r}, t) &= \vec{F}_i(\vec{k}_i, \omega_i)\cos(\vec{k}_i\cdot\vec{r} - \omega_i t) \\
        \vec{P}(\vec{r}, t) &= \vec{P}(\vec{k}_i, \omega_i)\cos(\vec{k}_i\cdot\vec{r} - \omega_i t) \\
        \vec{P}(\vec{k}_i, \omega_i) &= \chi(\vec{k}_i, \omega_i)\vec{F}_i(\vec{k}_i, \omega_i)
    \end{align}

    The lattice has thermal vibrations, quantized into phonons, causing fluctuations in $\chi$. The atomic displacements of a phonon
    can also be expressed as a plane wave, with wavevector and frequency $\vec{q}, \omega_0$:

    \begin{align}
        \vec{Q}(\vec{r}, t) = \vec{Q}(\vec{q},\omega_0)\cos(\vec{q}\cdot\vec{r}-\omega_0 t)
    \end{align}

    These phonons will perturb $\chi$. Assuming the characteristic electronic frequencies, which determine $\chi$ are
    much larger than $\omega_0$, $\chi$ can be assumed to be a function of $\vec{Q}$. At room temperature the
    amplitudes of the vibrations are small when compared to the lattice constant, meaning we can expand $\chi$ as
    a Taylor series of $\vec{Q}$:

    \begin{align}
        \chi(\vec{k}_i, \omega_i, \vec{Q}) = \chi_0(\vec{k}_i, \omega_i) + \frac{\partial\chi}{\partial\vec{Q}}_0\vec{Q}(\vec{r},t) + ...
    \end{align}

    where $\chi_0$ is the unperturbed susceptibility and the second term is the effect of the lattice wave. Knowing this, we
    can express the polarization of the medium with lattice vibrations:

    \begin{align}
        \vec{P}(\vec{r}, t, \vec{Q}) &= \vec{P}_0(\vec{r},t) + \vec{P}_{ind}(\vec{r}, t, \vec{Q}) \\
        \vec{P}_0 &= \chi_0(\vec{k}_i, \omega_i)\vec{F}_i(\vec{k}_i, \omega_i)\cos(\vec{k}_i\cdot\vec{r} - \omega_i t) \\
        \vec{P}_{ind}(\vec{r}, t, \vec{Q}) &= \frac{\partial\chi}{\partial\vec{Q}_0}\vec{Q}(\vec{r},t)
                                                \vec{F}_i(\vec{k}_i, \omega_i)\cos(\vec{k}_i\cdot\vec{r}-\omega_i t)
    \end{align}

    Such a simplistic description only includes interaction between TO phonons and EM waves, neglecting LO phonons, which
    can interact with EM waves indirectly, through macroscopic EM fields, but at this moment this is not a serious deficiency.

    \begin{align}
        \vec{P}_{ind}(\vec{r}, t, Q) &= \frac{\partial\chi}{\partial\vec{Q}_0}\vec{Q}(\vec{q},\omega_0)\cos(\vec{q}\cdot\vec{r}-\omega_0 t) \nonumber\\
                                                &\times\vec{F}_i(\vec{k}_i, \omega_i)\cos(\vec{k}_i\cdot\vec{r}-\omega_i t) \\
                    &= \frac{1}{2}\frac{\partial\chi}{\partial\vec{Q}_0}\vec{Q}(\vec{q},\omega_0)\vec{F}_i(\vec{k}_i, \omega_i) \nonumber\\
                    &\cdot \left( \cos((\vec{q} + \vec{k}_i)\cdot\vec{r} + (\omega_0 + \omega_i) t)
                                + \cos((\vec{q} - \vec{k}_i)\cdot\vec{r} + (\omega_0 - \omega_i) t)) \right)
    \end{align}

    $\vec{P}_{ind}$ contains two sinusoidal waves - a Stokes shifted ($\omega_S = \omega_0 - \omega_i, \vec{k}_S = \vec{k}_i - \vec{q}$) and an
    anti-Stokes shifted wave ($\omega_A = \omega_0 - \omega_i, \vec{k}_A = \vec{k}_i - \vec{q}$). This produces Stokes and anti-Stokes scattered
    light, with the difference in frequency from the original wave know as the Raman shift.

    Since in this case both frequency and wavevector are conserved, single-phonon raman scattering probes only zone-center phonons.
    Expanding the Taylor series we can easily move to multiple phonon scattering. For two phonon scattering we get combination
    and difference modes. If the two phonons are identical, then we observe overtone scattering. In this case there is no limit
    on the wavevector of the individual phonons (only that they need to be identical), meaning that overtone Raman probes the
    overall phonon density of states.

    The intensity of the Raman scattering depends on the polarization of the incident radiation, the scattered radiation and the
    types of phonons participating in the scattering.

    \begin{align}
        I_s \propto | \vec{e}_i \cdot \frac{\partial\chi}{\partial\vec{Q}}\vec{Q}(\omega_0)\cdot\vec{e}_s|^2
    \end{align}

    This approximates $\vec{q} = 0$ for single phonon scattering. $\frac{\partial\chi}{\partial\vec{Q}}$ is a third-rank tensor with
    complex components. Introducing $\vec{Q}_n = \vec{Q}/|\vec{Q}|$, a unit vector in the direction of the phonon displacement, we can
    define a complex second rank tensor,

    \begin{align}
        \hat{R} &= \frac{\partial\chi}{\partial\vec{Q}}\vec{Q}_n \\
        I_s &\propto | \vec{e}_i \cdot \hat{R} \cdot\vec{e}_s|^2
    \end{align}

    $\hat{R}$ is the Raman tensor, whose symmetry determines the symmetry of the material's Raman-active phonons. The symmetry of the
    Raman tensor depends on the symmetry of the medium and the active phonons.

\clearpage

\section{Femtosecond Laser Ablation}
\label{ap:Ablation}

    \subsection{Generation of femtosecond laser pulses}
            Femtosecond laser pulse generation is usually done using chirped pulse amplification. A mode-locked seed laser is used
        to generate a train of low-power femtosecond pulses, which are then temporally stretched, amplified, compressed and output
        from the laser system. This is necessary, because the final, compressed femtosecond pulse can have extremely high peak power,
        which would damage the amplification system\cite{harilal2014femtosecond}.
            The compression and stretching is done by using dispersion to cause different wavelengths of light to travel different distances.
        This is usually accomplished using either two prisms and a mirror or two diffraction gratings and a mirror, though, using engineered
        dispersion in optical fibers is also possible\cite{harilal2014femtosecond}.
            \hlr{Our femtosecond laser system, Femtosecond Oscillator TiF-100F by Avesta Project, is a Ti:Sapphire laser pumped by a Nd:YLF frequency
        doubled laser, emitting laser pulses at a central wavelength of $800~\si{nm}$, with pulse duration of $100~\si{fs}$, and repetition
        frequency of $80~\si{MHz}$}.

    \subsection{Ultrashort-pulse laser ablation}
            Laser ablation by ultrashort pulses, femto- and picosecond pulses, is a very efficient technique of patterning materials, because
        the short pulse length minimizes the influence of heat conduction on the ablated volume~--- keeping the ablation very localized and
        controlled.

            In metals and semiconductors having a large concentration of conduction band electrons, most of the light from the pulse is
        absorbed by conduction band electrons. The conduction band electrons thermalize within a timeframe of $10$fs-$1$ps, while thermalization
        between the electrons and the lattice is much slower, on the order of $1-100$ps, meaning that after the absorption of the laser pulse,
        we have a non-equilibrium state of a hot electron gas at temperature $T_e$ and a cold lattice at $T$. \cite{bauerle2013laser}

        The evolution of the temperature of the electron gas and the lattice can be described by the following heat equations:

        \begin{align}
            C_e \frac{\partial T_e}{\partial t} &= \nabla (\kappa_e \nabla T_e) - \Gamma_{e-ph}(T_e - T) + Q(x_\alpha, t) \label{ablation:e-heat} \\
            C \frac{\partial T}{\partial t} &= \nabla (\kappa \nabla T) + \Gamma_{e-ph}(T_e - T) \label{ablation:heat}
        \end{align}

        With $C_e, C$ as the heat capacities of the electron gas and the lattice. For a 1D approximation, the source term can
        be written as
        \begin{align}
            Q(z,t) = \alpha A I(t) \exp(-\alpha z)
        \end{align}

        For femtosecond pulses, heat conduction in within the lattice (first right-hand term of Eq.~\ref{ablation:heat}) can be ignored.
        Because the heat capacity of the electron gas is much smaller than that of the lattice, $C_e \ll C$,  the electron gas can
        be heated to very high transient temperatures.

        \begin{align}
            C_e &= C_0 T_e, \quad T_e \ll T_{Fermi} \equiv \frac{E_F}{k_B} \\
            C_0 &= \frac{\pi^2 N_e k_B}{2T_F} \\
            C &= const, \quad T > \theta_{Debye}
        \end{align}

        The non-equilibrium thermal conductivity of electrons can be approximated by

        \begin{align}
            \kappa_e = \kappa_e(T) \times \frac{T_e}{T}
        \end{align}

        Where $\kappa_e(T)$ is the normal, equilibrium, heat conductivity.

            For femtosecond pulses, the characteristic cooling time of the hot electron gas due to energy exchange with the lattice
        is larger than the pulse duration, $\tau_l \ll \tau_e \equiv \frac{C_e}{\Gamma_{e-ph}}$. For
        $t \ll \tau_e$ or $\Gamma_{e-ph}T_e \ll \frac{C_e T_e}{t}$, electron-phonon coupling can be ignored. Another reasonable approximation,
        considering the thermal diffusivity of electrons $D_e = \frac{\kappa_e}{C_e}$, $D_e\tau_l < \alpha^{-2}$, is to ignore heat
        conduction by electrons. Then Eq.~\ref{ablation:e-heat} simplifies to:

        \begin{align}
            \frac{1}{2}C_0\frac{\partial T_e^2}{\partial t} &= \alpha I_a \exp(-\alpha z) \\
            T_e(t) &= \left(T_0^2 + \frac{2\alpha\phi_a(t)}{C_0}\exp(-\alpha z)\right)^2 \\
            \phi_a(t) &= \int_0^t I_a(t')dt'
        \end{align}

        Where $T_0$ is the initial temperature.

        By the end of the pulse, $t = \tau_l$, we get:

        \begin{align}
            T_e(\tau_l) \approx \left(\frac{2\alpha\phi_a}{C_0}\right)^{\frac{1}{2}}\exp\left(-2\frac{\alpha z}{2}\right)
        \end{align}

        For times $t \geq \tau_l$, Equations \ref{ablation:heat} and \ref{ablation:e-heat}, with $Q = 0$ describe the evolution of the
        two systems. The electron gas then rapidly dumps all the energy to the lattice. Continuing to ignore heat conduction, the lattice
        temperature:

        \begin{align}
            T &\approx \frac{\alpha \phi_a}{C}\exp(-\alpha z) \\
            CT &= \int_0^{T_e} C_e(T_e')dT_e'
        \end{align}

        Significant ablation will occur if $CT \approx \Delta H_v$, where $\Delta H_v$ is the transition enthalpy. All of these approximations
        hold if $T_e \ll \frac{E_F}{k_B}$. The ablated depth is approximately

        \begin{align}
            \Delta h &= \frac{1}{\alpha}\ln\frac{\phi}{\phi_{th}}\\
            \phi_{th} &= \frac{\Delta H_v}{\alpha A}
        \end{align}

        This is a very crude approximation, that disregards energy transport by ballistic and diffusive electron propagation; lattice deformation,
        thermionic electron emission, etc... A more rigorous treatment would include lattice deformations caused by the heated electron gas. The
        deformation wave caused by the electron gas could cause the metal to fracture and ablate, without significant heating of the lattice itself.

\section{Discrete Dipole Approximation}
\label{ap:DDA}

    The discrete dipole approximation is a method of numerically simulating light scattering from arbitrarily shaped particles.
    The general idea of the method is to replace an arbitrarily shaped scatterer by a set of point dipoles and calculate the
    scattering by each dipole on its one plus the interaction between the dipoles. This makes calculations straightforward
    for scatterers of arbitrary geometries and compositions.

    The following derivation is based on the derivation found in Ref. \cite{yurkin2007discrete}. For it we assume
    non-magnetic materials $\mu = 1$ and $e^{-i\omega t}$ time dependence. For simplicity, the dielectric permittivity is assumed
    to be isotropic, i.e. scalar. Generalization to anisotropic scalars is generally straightforward.

    The general form of the integral equation describing the electric field inside the dielectric scatterer can be written as follows:
    \begin{align}
        \vec{E}(\vec{r}) = \vec{E}_{inc}(\vec{r}) &+ \int_{V \setminus V_0}d^3r' \hat{G}(\vec{r}, \vec{r}')\chi(\vec{r}')\vec{E}(\vec{r}')\nonumber\\
                        &+ \hat{M}(V_0, \vec{r}) - \hat{L}(\partial V_0, \vec{r})\chi(\vec{r})\vec{E}(\vec{r})
    \end{align}
    Where $\vec{E}_{inc}(\vec{r})$ is the incident field, $\vec{E}(\vec{r})$ is the total field at point $\vec{r}$.
    $\chi(\vec{r}) = \frac{\epsilon(\vec{r}) - 1}{4\pi}$. $V$ is the total volume, $V_0 \subset V$, $\vec{r} \in V_0\setminus \partial V_0$.

    $\hat{G}(\vec{r}, \vec{r}')$ is the free space dyadic Green's function:

    \begin{align}
        \hat{G}(\vec{r}, \vec{r}') &= \left(k^2\hat{I}+\hat{\nabla}\hat{\nabla}\right)\frac{e^{ikR}}{R} \\
                                    &= \frac{e^{ikR}}{R}\left(k^2\left(\hat{I}-\frac{\hat{R}\hat{R}}{R^2}\right)\right.
                                        \left.-\frac{1-ikR}{R^2}\left(\hat{I} -3 \frac{\hat{R}\hat{R}}{R^2}\right)\right)
    \end{align}

    where: $k = \frac{\omega}{c}$, $\vec{R} = \vec{r} - \vec{r}'$, $R = |\vec{R}|$, $\hat{R}\hat{R}$ is a dyadic $\hat{R}\hat{R}_{\mu\nu} = R_\mu R_nu$

    $\hat{M}$ is an integral associated with the finite exclusion volume $V_0$:

    \begin{align}
        \hat{M}(V_0, \vec{r}) = \int_{V_0}d^3r'\left(\hat{G}(\vec{r}, \vec{r}')\chi(\vec{r}')\vec{E}(\vec{r}')
                                - \hat{G}^s(\vec{r}, \vec{r}')\chi(\vec{r}')\vec{E}(\vec{r}')\right)
    \end{align}

    where $\hat{G}^s(\vec{r}, \vec{r}')$ is the static limit $(k \rightarrow 0)$ of $\hat{G}(\vec{r}, \vec{r}')$:

    \begin{align}
        \hat{G}^s(\vec{r}, \vec{r}')\chi(\vec{r}')\vec{E}(\vec{r}') = \hat{\nabla}\hat{\nabla}\frac{1}{R}
                                            = -\frac{1}{R^3}\left(\hat{I}-3\frac{\hat{R}\hat{R}}{R^2}\right)
    \end{align}

    $\hat{L}$ is the self-interaction dyadic:

    \begin{align}
        \hat{L}(\partial V_0, vec{r}) = -\oint_{\partial V_0} d^2 r' \frac{\hat{n}'\hat{R}}{R^3}
    \end{align}

    Where $\hat{n}'$ is an external normal to the surface of $V_0$, $\partial V_0$ at $\vec{r}'$. $\hat{L}$ is an always real, symmetric dyadic
    with trace equal to $4\pi$. $\hat{L}$ does not depend on the size of the volume, only on its shape. $\hat{M}$ depends on the size of the
    volume and approaches $0$ when the size of the volume decreases.

    The original integral equation is then discretized:

    \begin{align}
        V &= \bigcup^N_{i=1}V_i \\
        V_i \cap V_j &= 0, i \neq j
    \end{align}

    For simplicity, the volumes are general equal, and in the DDA are called dipoles. Assuming $\vec{r} \in V_i$ and $V_0 = V_i$, the first
    equation becomes:

    \begin{align}
        \vec{E}(\vec{r}) = \vec{E}_{inc}(\vec{r}) &+ \sum_{j\neq i} \int_{V_j} d^3 r'\hat{G}(\vec{r}, \vec{r}')\chi(\vec{r}')\vec{E}(\vec{r}') \nonumber\\
                                                &+ \hat{M}(V_i, \vec{r}) - \hat{L}(\partial V_i, \vec{r})\chi(\vec{r})\vec{E}(\vec{r})
    \end{align}

    This sum is exact. Next, we fix $\vec{r_i}$ in each $V_i$~--- its center. Then, for $\vec{r}=\vec{r}_i$, we can assume that

    \begin{align}
        \int_{V_j} d^3 r' \hat{G}(\vec{r}_i,\vec{r}')\chi(\vec{r}')\vec{E}(\vec{r}') &= V_j \hat{G}_{ij}\chi(\vec{r}_j)\vec{E}(\vec{r}_j) \\
        \hat{M}(V_j, \vec{r}_j) &= \hat{M}_i \chi(\vec{r}_i)\vec{E}(\vec{r}_i)
    \end{align}

    meaning that the integrals depend on the values of $\chi, \vec{E}$ at $\vec{r}_i$. Further, the integral equation can be written as

    \begin{align}
        \vec{E}_i &= \vec{E}_{i, inc} + \sum_{i\neq j}\hat{G}_{ij}V_j\chi_j\vec{E}_j + \left(\hat{M}_i -\hat{L}_i\right)\chi_i\vec{E}_i \\
        \vec{E}_j &= \vec{E}(\vec{r}_j) \\
        \vec{E}_{i, inc} &= \vec{E}_{inc}(\vec{r}_j) \\
        \chi_j &= \chi(\vec{r}_j) \\
        \hat{L}_j &= \hat{L}(\partial V_j, \vec{r}_j)
    \end{align}

    Generally, the subvolumes are assumed to be small enough that

    \begin{align}
        \vec{E}(\vec{r}) &= \vec{E}_i \\
        \chi(\vec{r}) &= \chi_i \\
        \vec{r} &\in V_i
    \end{align}

    meaning that

    \begin{align}
        \hat{M}_{i}^{approx} &= int_{V_i}d^3r'\left(\hat{G}(\vec{r}_i, \vec{r}') - \hat{G}^s(\vec{r}, \vec{r}')\right) \\
        \hat{G}_{ij}^{approx} &= \frac{1}{V_j}\int_{V_j}d^3r'\hat{G}(\vec{r}_i, \vec{r}')
    \end{align}

    next we apply a further approximation,

    \begin{align}
        \hat{G}_{ij}^{approx} = \hat{G}(\vec{r}_i, \vec{r}_j)
    \end{align}

    This assumption is equivalent to replacing the initial scattering volume by a set of point dipoles. It is possible
    to formulate the DDA with a weaker set of assumptions, but the greatly increases computational complexity.

    The DDA solves for exciting electric fields:

    \begin{align}
        \vec{E}_i^{exc} &= \left(\hat{I} + \left(\hat{L}_i - \hat{M}_i \right)\chi_i\right)\vec{E}_i = \vec{E}_i - \vec{E}_i^{self} \\
        \vec{E}_i^{self}&= \left(\hat{M}_i - \hat{L}_i \right)\chi_i\vec{E}_i
    \end{align}

    Where $\vec{E}_i^{sefl}$ is the field induced by the subvolume on itself. Then the original equation is equivalent to

    \begin{align}
        \vec{E}_i^{inc} = \vec{E}_i^{exc} - \sum_{j\neq i}\hat{G}_{ij}\hat{\alpha}_j\vec{E}_j^{exc}
    \end{align}

    where $\hat{\alpha}_i$ is the polarizability tensor:

    \begin{align}
        \hat{\alpha}_i &= V_i\chi_i\left(\hat{I} + \left(\hat{L}_i - \hat{M}_i\right)\chi_i\right)^{-1}
    \end{align}

    An equivalent formulation of the DDA solves for induced polarizations:

    \begin{align}
        \vec{P}_i = \hat{\alpha}_i\vec{E}_i^{exc} = V_i\chi_i\vec{E}_i \\
        \vec{E}_i^{inc} = \hat{\alpha}_i^{-1}\vec{P}_i - \sum_{j \neq i} \hat{G}_{ij}\vec{P}_j
    \end{align}

    This formulation turns out to be preferable for numerical simulations.

    Different formulations of the DDA use different approximations for the polarizability tensor $\hat{\alpha}$. The original
    formulation uses the Clausius-Mossoti polarizability:

    \begin{align}
        \hat{\alpha}_i = \hat{I}\alpha_i^{CM} = \hat{I}d^3\frac{3}{4\pi}\frac{\epsilon_i -1}{\epsilon_i + 2}
    \end{align}

    After determining the internal field, we can calculate the scattered fields and cross sections of the scatterer. The
    scattered fields obtained by taking the limit $r \rightarrow \infty$ of the integral in the initial equation, from which
    all of the DDA was derived:

    \begin{align}
        \vec{E}^{sca}(\vec{r}) &= \frac{e^{ikr}}{-ikr}\vec{F}(\vec{n}) \\
        \vec{F}(\vec{n}) &= -ik^3\left(\hat{I}
                            - \hat{n}\hat{n}\right)\sum_i \int_{V_i}d^3r'e^{-ik\vec{r}'\cdot\vec{n}}\chi(\vec{r}')\vec{E}(\vec{r}')
        \vec{n} &= \frac{\vec{r}}{r}
    \end{align}

    Knowing $\vec{F}(\vec{n})$, any other necessary scattering properties can be calculated. E.g. cross sections. For
    an incident plane wave:

    \begin{align}
        \vec{E}^{inc}(\vec{r}) = \vec{e}^0 e^{i\vec{k}\cdot\vec{r}}
    \end{align}

    The scattering cross section, $C_{sca}$ is:

    \begin{align}
        C_{sca} = \frac{1}{k^2}\oint d\Omega \left|\vec{F}(\vec{n})\right|^2
    \end{align}

    using internal fields, absorption and extinction cross sections:

    \begin{align}
        C_{abs} &= 4\pi k \sum_i \int_{V_i} d^3r' \Im(\chi(\vec{r}'))\left|\vec{E}(\vec{r}')\right|^2\\
        C_{ext} &= 4\pi k \sum_i \int_{V_i} d^3r' \Im\left(\chi(\vec{r}')\vec{E}(\vec{r}')\cdot(\vec{E}^{inc}(\vec{r}'))^*\right) \\
        &= \frac{4\pi}{k^2}\Re\left(\vec{F}(\frac{\vec{k}}{k})\cdot\vec{e}^{0*}\right)
    \end{align}

    These can be expressed in terms of internal fields:

    \begin{align}
        C_{abs} &= 4\pi k \sum_i V_i \Im(\chi_i)|\vec{E}_i|^2 = 4\pi k \sum_i \Im(\vec{P}_i\vec{E}_i^*) \\
        C_{ext} &= 4\pi k \sum_i \Im (\vec{P}_i\cdot\vec{E}_i^{inc*})
    \end{align}

    Most errors in the DDA are related to discretization errors, shape errors or the model used to describe the polarizability tensor.



\section{Finite Integration Technique}
\label{ap:FIT}
    The Finite Integration Technique (FIT) is a method for discretizing the Maxwell equations onto an arbitrary
    grid\cite{weiland2001discrete}.  Because the FIT deals with the integral forms of the Maxwell equations, it,
    Unlike the Finite-Difference Time-Domain (FDTD) methods, does not have any restrictions on the type of grid,
    other than that it be homeomorphic to a simplicial complex.

    For the simplicity of the following derivation\cite{rahimi2011finite}, we will assume the cells of the grid to be brick shaped. In this
    case the cell complex can be described as follows:

    \begin{align}
        \Omega &= [0, L_x]\times[0, L_y]\times[0, L_z] \\
        \Omega_{c,x} &= \{x_i, x_1, ..., x_m\}, x_i = \frac{L_x - 0}{m}*i \\
        \Omega_{c,y} &= ... \\
        \Omega_{c,z} &= ... \\
        \Omega_{s,x} &= \{s_0, s_1 ..., s_m-1\}, s_i = \frac{1}{2}(x_i + x_i+1) \\
        \Omega_{s,y} &= ... \\
        \Omega_{s_z} &= ...
    \end{align}

    These can be combined into 8 different three-dimensional grids, combining main ($c$) and staggered ($s$) grid points:

    \begin{align}
        \Omega_{t_x, t_y, t_z} &= \Omega_{t_x}\times\Omega_{t_y}\times\Omega_{t_z}\\
        t_i &= (c, s)
    \end{align}


    \begin{figure}
        \centering
        \includegraphics[width=0.5\linewidth]{figs/methods/FIT/directions.tikz}
        \caption{Schematic of used names for sides/directions of the unit cell}
        \label{fig:Dir_Int}
    \end{figure}


    The grid is specified by cell size and overall computational domain size.
    Simplest case is when $h_x = h_y = h_z$, a uniform grid.

    \begin{align}
        h_x = \frac{L_x}{m_x} \\
        h_y = \frac{L_y}{m_y} \\
        h_z = \frac{L_z}{m_z} \\
    \end{align}

    Discretizing of the Maxwell equations on this set of grids within the framework of the
    FIT is done starting with the integral form of the equations:

    \begin{align}
        \frac{\partial}{\partial t}\int\int_{A_p} \epsilon(\vec{r})\vec{E}(\vec{r},t)d\vec{A}
            &= \oint_{\partial A_p} \vec{H}(\vec{r}, t)d\vec{r} - \int\int_{A_p}\sigma(\vec{r})\vec{E}(\vec{r}, t)d\vec{A} \\
        \frac{\partial}{\partial t}\int\int_{A_p^*} \mu(\vec{r})\vec{H}(\vec{r}, t)d\vec{A}^*
            &= -\oint_{\partial A_p^*} \vec{E}(\vec{r},t)d\vec{r} - \int\int_{A_p^*}\sigma^*(\vec{r})\vec{H}(\vec{r},t)d\vec{A}^*\\
    \end{align}

    These equations a are then discretized on the staggered grid, with electric field calculated on the main
    grid and magnetic on the staggered one (superscript denotes timestep).

    \begin{align}
        \frac{\vec{E}_h^{n+1} - \vec{E}_h^{n}}{\tau}\int\int_{A_p}\epsilon(\vec{r})d\vec{A}
            &= \oint_{\partial A_p}\vec{H}_h^{n+\frac{1}{2}}(\vec{r})d\vec{r} - \vec{E}_h^{n+1}\int\int_{A_p}\sigma(\vec{r})d\vec{A} \\
        \frac{\vec{H}_h^{n+\frac{1}{2}} - \vec{H}_h^{n-\frac{1}{2}}}{\tau} \int\int_{A_p^*}\mu(\vec{r})d\vec{A}
            &= - \oint_{\partial A_p}\vec{E}_h^{n}(\vec{r})d\vec{r} - \vec{H}_h^{n+\frac{1}{2}}\int\int_{A_p}\sigma^*(\vec{r})d\vec{A}^*
    \end{align}

    Where effective permittivities and conductivities at points other than where they are defined are
    approximated by averaging the values from the closest available points.

    \begin{figure}[!ht]
        \centering
        \includegraphics[width=0.5\linewidth]{figs/methods/FIT/ex_int.tikz}
        \caption{Surface of integration of a cell for $E_x$ component}
        \label{fig:Ex_Int}
    \end{figure}

    for the cell depicted in Figure. \ref{fig:Ex_Int}, we have

    \begin{align}
        d\vec{A} &= \vec{n}dA = \vec{e}_xdydz \\
        d\vec{r}_y &= \vec{t}dr = \vec{e}_ydy \\
        d\vec{r}_z &= \vec{t}dr = \vec{e}_zdz
    \end{align}

    which means that the above equations simplify to

    \begin{align}
        \oint_{\partial A_p} \vec{H}|^{n+\frac{1}{2}}_{C}(\vec{r})d\vec{r} &= \int_{C_1}H_y|_D^{n+\frac{1}{2}} - \int_{C_3} H_y|_T^{n+\frac{1}{2}}dy \\
        &+\int_{C_2}H_z|_{N}^{n+\frac{1}{2}}dz - \int_{C_4}H_z|_{S}^{n+\frac{1}{2}}dz \\
        &= H_y|_D^{n+\frac{1}{2}}\int_{C_1}dy - H_y|_T^{n+\frac{1}{2}} \int_{C_3} dy +H_z|_{N}^{n+\frac{1}{2}}\int_{C_2}dz - H_z|_{S}^{n+\frac{1}{2}}\int_{C_4}dz\\
        &= H_y|_D^{n+\frac{1}{2}}\Delta y - H_y|_T^{n+\frac{1}{2}} \Delta y +H_z|_{N}^{n+\frac{1}{2}}\Delta z - H_z|_{S}^{n+\frac{1}{2}}\Delta z\\
    \end{align}

    Based on this, the update equations for the components of $\vec{E}$ can be written as:

    \begin{align}
        E_x|_M^{n+1} &= \frac{1}{1+\tau\frac{\sigma_p}{\epsilon_p}}E_x|_M^n
            + \frac{\frac{\tau}{\sigma_p}}{1+\tau\frac{\tau\sigma_p}{\epsilon_p}}\left[\frac{H_z|_N^{n+\frac{1}{2}}-H_z|_S^{n+\frac{1}{2}}}{\Delta y}
            - \frac{H_y|_T^{n+\frac{1}{2}} - H_y|_D^{n+\frac{1}{2}}}{\Delta z}\right]\\
        E_y|_M^{n+1} &= \frac{1}{1+\tau\frac{\sigma_p}{\epsilon_p}}E_y|_M^n
            + \frac{\frac{\tau}{\sigma_p}}{1+\tau\frac{\tau\sigma_p}{\epsilon_p}}\left[\frac{H_x|_T^{n+\frac{1}{2}}-H_x|_D^{n+\frac{1}{2}}}{\Delta z}
            - \frac{H_z|_W^{n+\frac{1}{2}} - H_z|_E^{n+\frac{1}{2}}}{\Delta x}\right]\\
        E_z|_M^{n+1} &= \frac{1}{1+\tau\frac{\sigma_p}{\epsilon_p}}E_z|_M^n
            + \frac{\frac{\tau}{\sigma_p}}{1+\tau\frac{\tau\sigma_p}{\epsilon_p}}\left[\frac{H_y|_W^{n+\frac{1}{2}}-H_y|_E^{n+\frac{1}{2}}}{\Delta x}
            - \frac{H_x|_N^{n+\frac{1}{2}} - H_x|_S^{n+\frac{1}{2}}}{\Delta y}\right]
    \end{align}

    Using the same method, we can derive the update equations for the components of $vec{H}$:

    \begin{align}
        d\vec{A}^* &= \vec{n}dA^* = \vec{e}_xdydz \\
        d\vec{R}_y &= \vec{t}dr = \vec{e}_ydy \\
        d\vec{r}_z &= \vec{t}dr = \vec{e}_zdz
    \end{align}

    \begin{figure}
        \centering
        \includegraphics[width=0.5\linewidth]{figs/methods/FIT/hx_int.tikz}
        \caption{Surface of integration of a cell for $H_x$ component}
        \label{fig:Hx_Int}
    \end{figure}


    \begin{align}
        \oint_{\partial A_p} \vec{E}|_{C}^{n}(\vec{r})d\vec{r} &= \int_{C_1}E_y|_{T}^ndy
                - \int_{C_3}E_y|_{D}^{n}dy - \int_{C_2}E_z|_{N}^{n}dz + \int_{C_4}E_z|_S^ndz \\
                &= E_y|_T^n\int_{C_1}dy - E_y|_D^n\int_{C_3}dy - E_z|_N^n\int_{C_2}dz + E_z|_S^n\int_{C_4}dz\\
                &= E_y|_T^n\Delta y - E_y|_D^n\Delta y - E_z|_N^n \Delta z + E_z|_S^n \Delta z
    \end{align}

    \begin{align}
        H_x|_M^{n+\frac{1}{1}} &= \frac{1}{1 + \tau\frac{\sigma_p^*}{\mu_p}}H_x|_M^{n-\frac{1}{2}} + \frac{\frac{\tau}{\sigma_p^*}}{1+\tau\frac{\tau\sigma_p^*}{\mu_p}}\left[ \frac{E_y|_T^n - E_y|_D^n}{\Delta z} - \frac{E_z|_N^n - E_z|_S^n}{\Delta y}\right]\\
        H_y|_M^{n+\frac{1}{1}} &= \frac{1}{1 + \tau\frac{\sigma_p^*}{\mu_p}}H_y|_M^{n-\frac{1}{2}} + \frac{\frac{\tau}{\sigma_p^*}}{1+\tau\frac{\tau\sigma_p^*}{\mu_p}}\left[\frac{E_z|_T^n - E_z|_D^n}{\Delta x} - \frac{E_x|_N^n - E_x|_S^n}{\Delta z}\right]\\
        H_z|_M^{n+\frac{1}{1}} &= \frac{1}{1 + \tau\frac{\sigma_p^*}{\mu_p}}H_z|_M^{n-\frac{1}{2}} + \frac{\frac{\tau}{\sigma_p^*}}{1+\tau\frac{\tau\sigma_p^*}{\mu_p}}\left[\frac{E_x|_T^n - E_x|_D^n}{\Delta y} - \frac{E_y|_N^n - E_y|_S^n}{\Delta x}\right]
    \end{align}

    The main advantage of using FIT over FDTD is that it is not tied the geometry of the grid~--- it is easier to optimize the geometry of the grid to the
    geometry that is being studied.


\clearpage
