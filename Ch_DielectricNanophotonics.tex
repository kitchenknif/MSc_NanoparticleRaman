% !TeX spellcheck = english
% !TEX root = thesis.tex
\section{Dielectric Nanophotonics}
\label{ch:Intro}

    \subsection{\hlb{Crystalline silicon as the material of choice for Dielectric Nanophotonics}}
            Crystalline silicon has been almost ubiquitously chosen as the material of choice for dielectric nanophotonics.
        There are several reasons for this. First, the material parameters are suitable for visible and infrared nanophotonic devices~---
        Crystalline silicon has a high refractive index, $n \approx 3.5$ at visible and IR wavelengths\cite{li1980refractive}, giving high contrast
        with air, and therefore good field confinement for resonant particles at those wavelengths\cite{mie1908beitrage, dmitriev2016resonant}. Also
        a very important fact, distinguishing crystalline silicon from amorphous silicon, is the near-zero absorbtion at visible wavelengths, meaning
        that crystalline silicon nanoparticles are nearly lossess at visible and IR wavelengths, making any potential nanophotonic devices very
        efficient and not prone to thermal dissipation~--- something that plagues their plasmonic counterparts.
            Silicon, being the basis of almost all modern electronics, can easliy boast a highly developed ecosystem of fabrication and processing
        technologies, making it very easy (if expensive) to fabricate almost any required structure.
            \hlr{Last, but not least, silicon is a bio-compatible material, meaning it is easy to design nanophotnoic devices for medical applications.}

            \hlr{TIE IN REQUIRED} Raman scattering of light is an important electromagnetic effect\cite{hayes2012scattering} that has a large number of applications:
        sensing\cite{moskovits1985surface}, optical amplification\cite{islam2004wideband}, lasing\cite{pask2003design}.
        Traditionally, enhancement of Raman scattering for SERS applications has been mostly delegated to metallic nanoparticles, but
        recent studies of high-index subwavelength nanoparticles have paved the way for all-dielectric resonant nanophotonic devices, incduling
        the possibility of enhancing Raman scattering, including intristic Raman from the nanoparticles.

    \subsection{Analytical Models}
        \subsubsection{\hlb{Mie-type resonances of dielectric nanoparticles}}
        \subsubsection{\hlb{Raman scattering from crystalline materials}}
        
    \subsection{Numerical Models}
        \subsubsection{\hlb{Discrete Dipole Approximation}}
        \subsubsection{Finite Element Method}
        \subsubsection{Method of Moments}
        \subsubsection{Bondary Element Method}
        \subsubsection{\hlb{Finite Integration Technique}}
        \subsubsection{Finite Difference Time Domain}

    \subsection{\hlb{Fabrication of Dielectric Nanoparticles}}
            There are several techniques to fabricate dielectric nanoparticles. They can be ordered in terms of level of control over the size
        of the particles and the options of building patterned nanostructures for the nanoparticles. Complexity and cost of the processes generally
        correlate with the level of control provided by the different techniques.
            The main techniques can be grouped into: chemical synthesis\cite{shi2012new}, thin-film dewetting\cite{abbarchi2014wafer},
        laser ablation-based\cite{zywietz2014laser} and lithographic methods.

        \subsubsection{\hlb{Chemical synthesis}}
                Chemical vapor deposition of silicon from disilane gas has been used to fabricate silicon nanoparticles, ($Si_2H_6 \rightarrow 2Si + 3H_2$ at
            high temperatures) achieving polycrystalline spherical particles\cite{shi2012new}. Monodisperse colloidal particles have been fabricated from trisilane ($Si_3H_8$)
            at high temperature in n-Hexane\cite{shi2013monodisperse}. Sizes of the particles were controlled by changing the concentration of the trisilane and the temperature of the
            reaction. The main disadvantages of these types of methods are the porosity and high hydrogen concentration of the particles and necessity of
            further processing if ordered nanostructures are required.

        \subsubsection{\hlb{Thin-film dewetting}}
                Thermal dewetting of thin films can be used to create arrays of nanoparticles, with size and phase controlled by the thickness of the intial
            film and temperature of the process\cite{abbarchi2014wafer}. If fabricated from an unpatterned film, the particles will be unordered~--- producing ordered particles is this
            way requires an additional patterning of the the film, usually by lithographic processes, which increases the complexity of the process and the cost.

        \subsubsection{\hlb{Laser ablation-based methods}}
                Laser ablation by focused ultra-short pulses can also be used to fabricated nanoparticles by heating up the irradiated area to eject material into
            spherical particles deposited either near the irradiated area\cite{kuznetsov2012magnetic} or transferred to another substrate\cite{zywietz2014laser}. Control of the beam sport, fluency and donor material
            structure and be used to control individual particle size and the number of particles fabricated from a single pulse (down to one particle in best-case
            scenario). The ultra-short pulses can also be used to change the phase of already fabricated structures by means of light-induced thermal annealing.

                The shortcomings of the current methods are that they often require additional annealing to crystallize the fabricated nanoparticles and that
            method presented in \cite{zywietz2014laser} is limited to transferring nanoparticles to transparent substrates.

        \subsubsection{\hlb{Lithographic methods}}
                Using electron beam lithography and reactive ion etching, one can pattern substrates with high-quality nanocylinders, with near-perfect
            control of both size and positioning of the structures\cite{bakker2015magnetic}. This give great flexibility in terms of fabricating ordered nanostructures, but at the
            cost of extreme complexity and high costs.

    \subsection{Experimental Characteriztion Methods}
    \label{sec:SEM}
            The geometrical parameters of the fabricated particles were measured by Scanning Electron Microscopy (SEM),
        Transmission Electron Microscopy (TEM).

        \subsubsection{\hlb{Scanning Electron Microscopy}}
                Scanning electron microscopy (SEM, Carl Zeiss, Neon 40), registering backscattered electrons, was used
            determine the geometrical parameters of the nanoparticles. In particular, that the particles possess \hlr{axial}
            symmetry along the substrate normal (Fig.~\ref{fig:Crystallinity}B).

        \subsubsection{\hlb{Transmission Electron Microscopy}}
                We used specimen grids (3-mm-diameter, 200-mesh copper grids, coated on one side with a 20-nm-thick film
            of amorphous carbon) to collect nanoparticles ablated from the a-Si:H film. The size, structure, and composition
            of the collected nanoparticles were determined using bright and dark field TEM imaging, see the inset in
            Fig.~\ref{fig:Crystallinity}B. The analysis of the electron diffraction pattern from several nanoparticles shows clear maxima,
            corresponding to certain crystalline planes (Fig.~\ref{fig:Crystallinity}C). \hlr{Because the specimen grids were uneven, the
            nanoparticles were deposited at different angles to the substrate meaning that} TEM imaging also provides information
            on the oblateness of the particles along the direction perpendicular to the substrate surface, giving
            the average ellipticity about $a_{\parallel}/a_{\perp}\approx$1.12, where $a_{\parallel}$($a_{\perp}$) is
            the particle semi-major (semi-minor) axis oriented parallel (perpendicular) to the surface of the substrate.

        \subsubsection{Optical characterization}
        \subsubsection{Scanning probe methods}

    \subsection{\hlb{Goals}}
            The goals of this project were to:
            \begin{itemize}
                \item develop a simple, single-stage femtosecond laser ablation method of frabricating crystallline silicon nanoparticles
                \item demonstrate resonant inelastic, Raman, scattering from the nanoparticles
            \end{itemize}

    %            \begin{figure}[h!]
    %    				\begin{center}
    %    					%\includegraphics[width=0.5\textwidth]{figs/intro/}
    %    				\end{center}
    %                    \label{fig:Lithography}
    %                    \caption{a. Schematic of lithographic process and b. SEM image of possible structure}
    %            \end{figure}
