% !TeX spellcheck = english
% !TEX root = thesis.tex
\section{Dielectric Nanophotonics}
\label{ch:DielectricNanophotoics}

    \subsection{Crystalline silicon as the material of choice for Dielectric Nanophotonics}
            Crystalline silicon has been almost ubiquitously chosen as the material of choice for dielectric nanophotonics.
        There are several reasons for this. First, the material parameters are suitable for visible and infrared nanophotonic devices~---
        Crystalline silicon has a high refractive index, $n \approx 3.5$ at visible and IR wavelengths\cite{li1980refractive}, giving high contrast
        with air, and therefore good field confinement for resonant particles at those wavelengths\cite{mie1908beitrage, dmitriev2016resonant}. Also
        a very important fact, distinguishing crystalline silicon from amorphous silicon, is the near-zero absorbtion at visible wavelengths, meaning
        that crystalline silicon nanoparticles are nearly lossess at visible and IR wavelengths, making any potential nanophotonic devices very
        efficient and not prone to thermal dissipation~--- something that plagues their plasmonic counterparts.
            Silicon, being the basis of almost all modern electronics, can easliy boast a highly developed ecosystem of fabrication and processing
        technologies, making it very easy (if expensive) to fabricate almost any required structure.
            \hlr{Last, but not least, silicon is a bio-compatible material, meaning it is easy to design nanophotnoic devices for medical applications.}

            Another interesting application dimension is related to the existence of and intristic Raman response in crystalline silicon. Raman
        scattering of light, and inelastic scattering phenomenon, is an important electromagnetic effect\cite{hayes2012scattering} that has a large number of applications:
        sensing\cite{moskovits1985surface}, optical amplification\cite{islam2004wideband}, lasing\cite{pask2003design}.
        Traditionally, enhancement of Raman scattering for SERS applications has been mostly delegated to metallic nanoparticles, but
        recent studies of high-index subwavelength nanoparticles have paved the way for all-dielectric resonant nanophotonic devices, including
        the possibility of enhancing Raman scattering, including intristic Raman from the nanoparticles.

    \subsection{Analytical Models}
        \subsubsection{Mie-type resonances of dielectric nanoparticles}
                An important problem for dieletric nanophotnoics is the scattering of eletromagnetic radiation by
            a homogeneous sphere. This problem has an analyticial solution, generally called Mie theory\cite{mie1908beitrage}.
            The solution of this problem utilizes the symmetry of the problem, and decomposes the scattered radiation into
            vector spherical harmonics. The result is a set of coeffeicients, $a_i, b_i$ describing the relative contributions
            of $a_i$, electric multipole resonances and $b_i$, magnetic multipole resonances into the final scattered field.

                Based on this, one can easily estimate the extinction, scattering and absorbtion cross-sections (for a detailed
                derivation, see Appendix \ref{ap:Mie}):

            \begin{align}
                C_{sca} &= \frac{2\pi}{k^2_m}\sum_{l=1}^\infty (2l +1)(|a_l|^2 + |b_l|^2)\\
                C_{ext} &= \frac{2\pi}{k^2_m}\Re\sum_{l=1}^\infty (2l +1)(a_l + b_l)\\
                C_{abs} &= C_{ext} - C_{sca}
            \end{align}

                The main limitation of the Mie solutions is their reliance on the symmetry of the scatterer~--- the solution can
            be generalized to layered spheres, layered  infinite cyllinders, but scatterers lacking symmetry cannot be accurately described
            by Mie-type solutions. In those cases, T-Matrix methods can be used, but they are outside the scope of this thesis.

        \subsubsection{Raman scattering from crystalline materials}
                Raman scattering of light from crystalline materials is inealstic scattering resulting from the interaction of optical phonons with
            the incident light. Raman scattering is a versatile method of probing the phonon stucture of the materials.
            A simplisitc classical model of Raman scattering is sufficient to demonstarte the effect and to properly predict many of the
            Raman scattering peaks of semiconductors\cite{peter2010fundamentals}. In the model, interaction between light and phonons
            is represented as a perturbation of the electric susceptibility $\chi$:

            \begin{align}
                \chi(\vec{k}_i, \omega_i, \vec{Q}) = \chi_0(\vec{k}_i, \omega_i) + \frac{\partial\chi}{\partial\vec{Q}}_0\vec{Q}(\vec{r},t) + ...
            \end{align}

            Where $Q$ represents the displacement of atoms in the lattice by phonons.
            Then the induced polarization by the incident light, perturbed by the phonons can be written as

            \begin{align}
                \vec{P}_{ind}(\vec{r}, t, \vec{Q}) &= \frac{\partial\chi}{\partial\vec{Q}_0}\vec{Q}(\vec{r},t)
                                                        \vec{F}_i(\vec{k}_i, \omega_i)\cos(\vec{k}_i\cdot\vec{r}-\omega_i t)
            \end{align}
            which contains both Stokes and anti-Stokes shifted components. The intensity of the Raman scattered light
            is dependant on the polarization of the incident light and the phonon structure of the material. All of this
            information can be condensed into a second rank Raman tensor:

            \begin{align}
                \hat{R} &= \frac{\partial\chi}{\partial\vec{Q}}\vec{Q}_n \\
                I_s &\propto | \vec{e}_i \cdot \hat{R} \cdot\vec{e}_s|^2
            \end{align}

            A more detailed derviation can be found in Appendix \ref{ap:Raman}.

    \subsection{Numerical Models}
            Since scattering from arbitrarily-shaped particles cannot be analytically described, use of numerical modelling methods
        is required to model these particles. A large number of numerical methods exist that can be used for these purposes.

        \subsubsection{Discrete Dipole Approximation}
            The discrete dipole approximation is a method of numerically simulating light scattering from arbitrarily shaped particles.
            The general idea of the method is to replace an arbirtarily shaped scatterer by a set of point dipoles and calculate the
            scattering by each dipole on its one plus the interaction between the dipoles. This makes calculations straightforward
            for scatterers of arbitrary geometries and compositions.


        \subsubsection{Finite Element Method}
                The Finite Element Method (FIT) is a numerical technique, first introduced in the 1940s\cite{courant1994variational}, to
            find approximate solutions to boundary-value problems. The technique has been used in many fields, from mechanical structure
            problems to electromagnetic field propagation.
                The idea of the method is to reaplace a continous domain by a number of sub-domains, in each of which the solution is
            approximated by interpolation functions. A linear system of algebraic equations is obtained by applying continuity boundary conditions
            on the internal sub-domains and problem-specific boundary conditions on the external subdomains. The system can then be solved
            by any solver for linear systems. The subdomains are generally chosen to be tetrahedra or hexahedra for three-dimensional solutions
            with the internal fields approximated by picewise polynomial functions, but, in general, the method has no limitations on the
            shape and size of the subdomains and the approximation functions inside each of the subdomains. This allows the method
            to accurately model complicated scattering geometries. The flip side of this flexibility~--- large space requirements for
            the system of equations. The method also often requires iterative refinement of the grid to achieve good convergence.

        \subsubsection{Finite Difference Time Domain}
                The Finitie Difference Time Domain (FDTD) method is very straightforward technique to solve the propagation of electromagnetic
            radiation, first developed by Yee\cite{yee1966numerical}. The method approximates both spatial and temporal derivates in the Maxwell's
            eqations by finite-difference expressions. The standard implementation does this on a staggered Cartesian grid, with the electric field
            defined on one sub-grid and the magnetic field defined on the second sub-grid. This results in a system of linear equations, where
            each time step is used to calculate the electric field on the first sub-grid from the magentic field on the second sub-grid, and then
            the magnetic field at the next time step from the electric field, and so on, until a termination condition is reached.
                The main advantage of this method is its simplicity~--- no matrix solvers required, possibility for very efficient implementations.
            The disadvantage is that this simplicity makes the method very rigid in terms of time step and spatial step choices for good convergence,
            to mitigate staircase-type errors caused by inefficient mapping of complex scatterer geometry onto a Cartesian grid.

        \subsubsection{Finite Integration Technique}
                The Finite Integration Technique (FIT)\cite{wieland1977discretization}, is idealogically similar to the FDTD method, but works with the
            integral formulation of Maxwell's equations instead of the derivative formulation used by the FDTD. This grants the method more flexibility
            in grid geometry, allowing for more accurate modelling of complicated scatters. The accuracy of the method, like with FDTD,
            is still highly dependent on the choice of grid size and time step.
                A detailed derivation for a simplistic case, with a Cartesian grid, can be found in Appendix \ref{ap:FIT}.

    \subsection{Fabrication of Dielectric Nanoparticles}
            There are several techniques to fabricate dielectric nanoparticles. They can be ordered in terms of level of control over the size
        of the particles and the options of building patterned nanostructures for the nanoparticles. Complexity and cost of the processes generally
        correlate with the level of control provided by the different techniques.
            The main techniques can be grouped into: chemical synthesis\cite{shi2012new}, thin-film dewetting\cite{abbarchi2014wafer},
        laser ablation-based\cite{zywietz2014laser} and lithographic methods.

        \subsubsection{Chemical synthesis}
                Chemical vapor deposition of silicon from disilane gas has been used to fabricate silicon nanoparticles, ($Si_2H_6 \rightarrow 2Si + 3H_2$ at
            high temperatures) achieving polycrystalline spherical particles\cite{shi2012new}. Monodisperse colloidal particles have been fabricated from trisilane ($Si_3H_8$)
            at high temperature in n-Hexane\cite{shi2013monodisperse}. Sizes of the particles were controlled by changing the concentration of the trisilane and the temperature of the
            reaction. The main disadvantages of these types of methods are the porosity and high hydrogen concentration of the particles and necessity of
            further processing if ordered nanostructures are required.

        \subsubsection{Thin-film dewetting}
                Thermal dewetting of thin films can be used to create arrays of nanoparticles, with size and phase controlled by the thickness of the intial
            film and temperature of the process\cite{abbarchi2014wafer}. If fabricated from an unpatterned film, the particles will be unordered~--- producing ordered particles is this
            way requires an additional patterning of the the film, usually by lithographic processes, which increases the complexity of the process and the cost.

        \subsubsection{Laser ablation-based methods}
                Laser ablation by focused ultra-short pulses can also be used to fabricated nanoparticles by heating up the irradiated area to eject material into
            spherical particles deposited either near the irradiated area\cite{kuznetsov2012magnetic} or transferred to another substrate\cite{zywietz2014laser}. Control of the beam sport, fluency and donor material
            structure and be used to control individual particle size and the number of particles fabricated from a single pulse (down to one particle in best-case
            scenario). The ultra-short pulses can also be used to change the phase of already fabricated structures by means of light-induced thermal annealing.

                The shortcomings of the current methods are that they often require additional annealing to crystallize the fabricated nanoparticles and that
            method presented in \cite{zywietz2014laser} is limited to transferring nanoparticles to transparent substrates.

        \subsubsection{Lithographic methods}
                Using electron beam lithography and reactive ion etching, one can pattern substrates with high-quality nanocylinders, with near-perfect
            control of both size and positioning of the structures\cite{bakker2015magnetic}. This give great flexibility in terms of fabricating ordered nanostructures, but at the
            cost of extreme complexity and high costs.

    \subsection{Experimental Characteriztion Methods}
        \subsubsection{Electron Microscopy}
        \label{sec:ElectronMicroscopy}
                Electron microscopy is a very high-resolution, down to single nanometer resolutions, method to probe geometrical and
            material parameters of structures. The structure is probed by a focused electron beam, measuring either scattered or transmitted
            primary electrons, secondary electrons or emitted photons. Because of chrage accumulation caused by the electron beam, samples
            usually need to be conductive to facilitate charge draining, or else the resulting signal will be distorted.
                The two main methods used are scanning electron microscopy (SEM) and transmission electron microscopy (TEM).

            \paragraph{Transmission Electron Microscopy}

            \paragraph{Scanning Electron Microscopy}

        \subsubsection{Optical characterization}
        \label{sec:OpticalCharacterization}
            \paragraph{Elastic Scattering}
            \paragraph{Inelastic Scattering}

        \subsubsection{Scanning probe methods}
        \label{sec:SPM}
            \paragraph{Near-field Scanning Optical Microscopy}


    \subsection{Goals}
            The project was devoted to the fabrication as well as the characterization of elastic and inelastic scattering from resonant crystalline
        silicon nanoaprticles.

            The goals of this project were to:
        \begin{itemize}
            \item Implement the Discrete Dipole Approximation method to calcualte scattering cross-sections of dielectric particles
            \item Develop a simple, single-stage femtosecond laser ablation method of frabricating crystallline silicon nanoparticles
            \item Carry out optical characterization through single-particle polarization-resolved scattering experiments
            \item Determine the sizes of the particles based on the spectral positions of the Mie-type resonances of the particles
            \item Perform single-particle Raman spectroscopy experiments to prove crystallinity
            \item Study the  influence of Mie resonances on Raman scattering from particles.
        \end{itemize}
